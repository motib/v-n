\documentclass[11pt]{article}
\usepackage{mathpazo}
\usepackage{url}
\usepackage[pdftex]{graphicx}
\usepackage{verbatim}

\newcommand{\vn}{\textsc{VN}}
\newcommand{\jf}{\textsc{jflap}}
\newcommand{\dt}{\textsc{dot}}
\newcommand{\spn}{\textsc{Spin}}
\newcommand{\prm}{\textsc{Promela}}
\newcommand{\erg}{\textsc{Erigone}}
\newcommand{\p}[1]{\texttt{#1}}
\newcommand{\bu}[1]{\textsf{#1}}

\textwidth=15cm
\textheight=21cm
\topmargin=0pt
\headheight=0pt
\oddsidemargin=1cm
\headsep=0pt
\renewcommand{\baselinestretch}{1.1}
\setlength{\parskip}{0.20\baselineskip plus 1pt minus 1pt}
\parindent=0pt

\title{\vn{} - Visualization of Nondeterminism\\
User's Guide\\\mbox{}\\\large{Version 3.2.3}}
\author{Mordechai (Moti) Ben-Ari\\
Department of Science Teaching\\
Weizmann Institute of Science\\
Rehovot 76100 Israel\\
\textsf{http://stwww.weizmann.ac.il/g-cs/benari/}}
%\date{}
\begin{document}

\maketitle
\thispagestyle{empty}

\vfil

\begin{center}
Copyright (c) 2006-9 by Mordechai (Moti) Ben-Ari.
\end{center}

Permission is granted to copy, distribute and/or modify this document
under the terms of the GNU Free Documentation License, Version 1.2 or
any later version published by the Free Software Foundation; with
Invariant Section ``Introduction,'' no Front-Cover Texts, and no
Back-Cover Texts. A copy of the license is included in the file
\p{fdl.txt} included in this archive.

\newpage

\section{Introduction}

\vn{} is a tool for studying the behavior of nondeterministic finite
automata (NDFA). It takes a description of an automaton and generates a
nondeterministic program; the program can then be executed randomly or
guided interactively. The automaton and the execution path are
graphically displayed.

\vn{} is written in Java for portability. It is based on other software
tools:

\begin{itemize}
\item The program generated from the automaton is written in \prm{}, the
language of the \spn{} model checker. See: M. Ben-Ari.
\textit{Principles of the Spin Model Checker}. Springer, 2008. The
program is run using the \erg{} model checker, which also uses the
\prm{} language. \url{http://code.google.com/p/erigone/}.

\item The graphical description of the automaton and path are created in
the \dt{} language and layed out by the \dt{} tool. Graphs in PNG format
are created and are then displayed within \vn{}. \dt{} is part of the
\textsc{Graphviz} package. \url{http://graphviz.org/}.
\end{itemize}

The \vn{} software is copyrighted according the the GNU General Public
License. See the files \p{copyright.txt} and \p{gpl.txt} in the archive.

The \vn{} webpage is \url{http://code.google.com/p/v-n/}.

\textbf{Acknowledgement:} Michal Armoni assisted in the design of \vn{}.

\section{Installation and execution}

\begin{itemize}
\item Install the Java SDK or JRE (\url{http://java.sun.com}).
\textbf{\vn{} needs Java 1.5 at least.}

\item For Windows: download the \vn{} installation file called \p{vn-N.exe},
where \p{N} is the version number, and execute the installation file.
For other systems, you will have to build \vn{}, and download and install
\erg{} and \dt{}.

\item The installation will create the following subdirectories: \p{docs} for the
documentation; \p{vn} for the source files; \p{bin} for the libraries and executables
for \erg{} and \dt{}; \p{txts} for the text files
(help, about and copyright); and \p{examples} for example programs.

\item To run \vn{}, execute the command \p{javaw -jar vn.jar}. An
optional argument names the file containing a representation of an
automaton. A batch file \p{run.bat} is supplied which contains this
command.

\item Configuration data for \vn{} is given in the source file \p{Config.java}, 
as well as in the file \p{config.cfg}. The latter is reset to its default values
if it is erased.

\item To rebuild \vn{}, execute \p{build.bat}, which will compile all the source
files and create the file \p{vn.jar} with the manifest file.
\end{itemize}

\section{Interacting with \vn{}}

The display of \vn{} is divided into three scrollable panes. Messages
from \vn{} are displayed in the text area at the bottom of the screen.
Graphs of automata are displayed in the left-hand pane and graphs of the
paths are displayed in the right-hand pane. The size of the panes is
adjustable and the small triangles on the dividers can be used to
maximize a pane. Above the panes is another text area called the path
area where execution paths of the automaton are displayed. Interaction
with VN is through a toolbar. All toolbar buttons have mnemonics
(Alt-character).

\begin{figure}[htbp]
\begin{center}
\includegraphics[width=.8\textwidth,keepaspectratio=true]{vn.png}
\end{center}
\end{figure}
 
\begin{description}

\item[\bu{Open}] Brings up a file selector; select a \p{jff} file and
the automaton described in the file will be displayed.

\item[\bu{Edit}] Invokes the grapical editor on the current automaton or
on an empty automaton.

\item[\bu{String or length}] Enter an input string for the automaton in
this text field. See below on entering a length.

\item[\bu{Generate}] Once an automaton has be opened and an input string
entered in the text field, selecting \bu{Generate} will create a program
to execute the automaton.

\item[\bu{Random}] This will execute the program for the automaton with
random resolution of nondeterminism. The sequence of states will appear
in the text area below the toolbar, along with an indication if the
input string was accepted or rejected.

\item[\bu{Create}] This will execute the program for the automaton,
resolving nondeterminism interactively. Deterministic choices will be
made automatically. If a potential nondeterministic choice cannot be
taken, it will be displayed in square brackets and cannot be selected.
\bu{Quit} terminates the execution. Keyboard shortcuts: \bu{Tab} moves
between buttons and \bu{Space} or \bu{Enter} selects the highlighted
button.

\item[\bu{Find}] This searches for an accepting computation of the
automaton.

\item[\bu{Next}] This searches for the next accepting computation of the
automaton. When no more accepting computations exist, the number of
accepting computations is displayed in the path area.

\item[\bu{DFA}] See below.

\item[\bu{Options}] Displays a dialog for choosing the size of the
graphs (small, medium, large) and whether the highlighting of the paths
will be in color or bold. When you select \bu{OK} the changes will be
saved, along with the directory of the current open file.

\item[\bu{Help}] Displays a short help file.

\item[\bu{About}] Displays copyright information about the software.

\item[\bu{Exit}] Exit the application.
\end{description}

Once a computation has been found , the \emph{full path} is displayed in
the right-hand pane, while the \emph{highlighted path} is displayed in
the automaton in the left-hand pane. The difference is that the full
path includes multiple visits to the same state, while the highlighted
path within the automaton will, of course, include just one state for
all occurences.

\noindent\textbf{Searching for all inputs that are accepted}

If you enter an integer value in the field \bu{String or length}, each
selection of \bu{Next} searches for an accepting computation for
\emph{any} input string of this length. When no more accepting
computations are found, the set of strings that are accepted is
displayed in path area. For the automaton \p{ndfa.jff} supplied in the
archive and with input length six, there are seven accepting
computations for the four inputs: \p{aaaaaa, aaaabb, aaabbb, aabbbb}.

\noindent\textbf{Equivalence classes of deterministic finite automata}

If the automaton is deterministic, \vn{} can display the partition of
the set of input strings of a given length into the equivalence classes
associated with each state.\footnote{This functionality was inspired by
\bu{ProofChecker}:\\
\hspace*{2cm}\url{http://research.csc.ncsu.edu/accessibility/ProofChecker/index.html}.}
Enter a length in the text field and select \bu{Generate}. Now select
\bu{DFA} repeatedly; for each state, starting from the first state, the
set of input strings ``accepted'' by that state is displayed. When
\bu{DFA} has been selected for all states, the set of equivalence
classes is displayed in the right-hand pane. \bu{Generate} can
be selected at any time to reset to the first state.

For the file \p{dfa.jff} in the \p{examples} directory, the equivalence classes
of length~$4$ are:
\begin{verbatim}
q0: [bbaa, baba, abba, aaaa]
q1: [bbab, babb, abbb, aaab]
q2: [bbba, baaa, abaa, aaba]
q3: [bbbb, baab, abab, aabb]
\end{verbatim}

\section{The graphical editor}

The input to \vn{} is a representation of an NDFA in \textsc{XML}. The
representation is that of the \jf{} software tool for studying formal
languages and automata. Therefore, \jf{} can be used as an external
graphics editor together with \vn{}. In addition, \vn{} contains a
graphics editor that is compatible with the \jf{} file format.

The editor has two \emph{tools}: one for creating states and one for
creating transitions. Click on the buttons \bu{State} or \bu{Transition}
to select one of the tools. Initially, the state tool is active. Other
buttons are: \bu{Save as} for saving the NDFA to another file and for
the initial save of a new NDFA. \bu{Quit} terminates the editor without
saving the file and \bu{Exit} saves the file and terminates.

When the save tool is active, clicking on the canvas will create a new
state. Right-clicking on the state will bring up a menu to designate the
state as initial and/or final, as well as to delete the state.

When the transition tool is active, a transition is created by pointing
the cursor at a state, pressing the left mouse button, moving to another
state and releasing the button. A self-loop can be created simply by
clicking on a state. The default symbol for the transition is \p{L} for
a $\lambda$-transition. Right-clicking on the symbol will bring up a
text field where a new symbol can be entered. If you delete the symbol,
the transition itself will be deleted when \bu{OK} is clicked.

The left mouse button can be used to drag and drop states or transitions
to reformat the NDFA. The state coordinates will be remembered if the
automaton is closed and later re-opened.

\section{Files}
The different phases of processing \vn{} communicate through files. Here is a
list of the extensions of these files:
\begin{itemize}
  \item[\p{jff}] XML representation of an automaton
  \item[\p{pml}] Promela source generated from the automaton
  \item[\p{pth}] Path resulting from running \spn{} on the \p{pml} file
  \item[\p{dot}] Graphics file describing automaton and path
  \item[\p{png}] \textsc{PNG} graphics file after layout by \dt{}
\end{itemize}

\section{Software structure}

The source code can be downloaded from GoogleCode:
\url{http://code.google.com/p/v-n/}. If you wish to use obtain source
code from the \textsc{Subversion} repository, please note that the
current code is in the branch named \p{vne}; the trunk represents an
older version that will not be maintained.

\p{VN} is the main class and contains declarations
of the GUI components and the event handler for all the buttons.

\p{Config} contains constants and properties.

\p{Options} displays a dialog for changing the size and highlight of the
graphs.

\p{DisplayFile} displays the files for \bu{About} and \bu{Help}
in a \p{JFrame}.

\p{JFFFileFilter} is used with a \p{JFileChooser} to open a \p{jff} file.

\p{ImagePanel} displays the \textsc{PNG} files.

\p{GenerateSpin} generates the \prm{} program from the automaton.

\p{ReadXML} reads the XML description of the automaton from the \p{jff}
file and stores the data in \p{states} and \p{transitions}. These types
are defined in classes \p{State} and \p{Transition}.

\p{ReadPath} reads the path data printed by the \prm{} program and
stores it in \p{pathStates} and \p{pathTransitions}.

\p{WriteGraph} writes the automaton and paths in \dt{} format and forks
a process to run \dt{} to layout the graph.

\p{RunSpin} forks a process to execute \spn{}. In interactive mode, it
reads the \prm{} program so that the \p{JOptionPane} used for selecting
among nondeterministic choices can display actual source code. For
\bu{Find} and \bu{Next} processes are forked to run the C compiler and
\p{pan}. For \bu{Next}, \p{pan} is run with the \p{-c} argument.

The graphics editor is in the subdirectory \p{editor}. It was adapted
from the \textsc{Autosim} software by Carl Burch:
\url{http://ozark.hendrix.edu/~burch/proj/autosim/index.html}
\end{document}
